\documentclass{article}
\usepackage{graphicx}
\usepackage{listings}
\usepackage{amsmath}
\usepackage{float} % Pour placer les images exactement où elles sont spécifiées

\renewcommand\lstlistingname{Code source}
\renewcommand\lstlistlistingname{Liste des codes sources}

\usepackage{xcolor}
\definecolor{codegreen}{rgb}{0, 0.5, 0} % Vert pour les commentaires
\definecolor{codegray}{rgb}{0.5, 0.5, 0.5} % Gris pour les numéros de ligne
\definecolor{codepurple}{rgb}{0.58, 0, 0.82} % Violet pour les mots-clés
\definecolor{backcolour}{rgb}{0.95, 0.95, 0.92} % Gris clair pour l'arrière-plan
\definecolor{codeorange}{rgb}{0.85, 0.33, 0.1} % Orange pour les chaînes de caractères
\definecolor{codeblue}{rgb}{0, 0, 1} % Bleu pour les variables

\lstdefinestyle{mystyle}{
    backgroundcolor=\color{backcolour},   
    commentstyle=\color{codegreen},
    keywordstyle=\color{codepurple},
    numberstyle=\tiny\color{codegray},
    stringstyle=\color{codeorange},
    basicstyle=\ttfamily\footnotesize,
    breakatwhitespace=false,         
    breaklines=true,                 
    captionpos=b,                    
    keepspaces=true,                 
    numbers=left,                    
    numbersep=5pt,                  
    showspaces=false,                
    showstringspaces=false,
    showtabs=false,                  
    tabsize=2,
    xleftmargin=10pt,
    morekeywords={xlim,ylim,legend,grid,title,axis,label}
}

\lstset{style=mystyle}

\sloppy
\definecolor{lightgray}{gray}{0.5}
\setlength{\parindent}{0pt}

\begin{document}

\begin{titlepage}
    \centering
    \includegraphics[width=0.3\textwidth]{EILCO-LOGO-2022.png}\par\vspace{1cm}
    {\scshape\LARGE école d'Ingénieurs du Littoral-Côte-d'Opale\par}
    \vspace{1cm}
    {\scshape\Large TP : Réseaux industriels et supervision\par} 
    \vspace{1.5cm}
    {\huge\bfseries  Interface Pygame\par}
    \vspace{0.5cm}
    \includegraphics[width=0.4\textwidth]{logo.png}\par
    \vspace{1.5cm}
    {\Large\itshape\textbf{Auteur:} Fono Colince\par}
    \vfill
    \begin{minipage}{0.4\textwidth}
        \begin{flushleft} \large
            \emph{Supervisé par:}\\
            Mr. \textsc{Pierre Chatelain} \\
        \end{flushleft}
    \end{minipage}
    \begin{minipage}{0.4\textwidth}
        \begin{flushright} \large
            \emph{Date:} \\
            \today
        \end{flushright}
    \end{minipage}
    \vfill
    \vspace{1cm}
\end{titlepage}

\clearpage

\tableofcontents  % Table des matières

\clearpage
% ------------------------------------------------------------------------------------

\section{Introduction}
\begin{quote}
    Dans ce TP, vous allez utiliser un simulateur d'automate qui simulera un automatisme industriel. Vous allez ensuite communiquer avec cet automate en utilisant le protocole Modbus pour produire une supervision.
\end{quote}

\section{ interface personnelle}

\section{Pendant le cycle de remplissage}

\section{Pendant le cycle de vidange}

\section{Annexe}
\subsection{La liste des icônes créé}


\subsection{Un diagramme des classes}

\subsection{ Vos codes sources}

\lstinputlisting[
    language=python,
    caption={.},
    label={lst:heaviside},
    mathescape=true,
    firstline=1,
    lastline=210 
]{../liquide.py}


\end{document}
    
